% Options for packages loaded elsewhere
\PassOptionsToPackage{unicode}{hyperref}
\PassOptionsToPackage{hyphens}{url}
%
\documentclass[
  english,
  man]{apa7}
\usepackage{amsmath,amssymb}
\usepackage{lmodern}
\usepackage{ifxetex,ifluatex}
\ifnum 0\ifxetex 1\fi\ifluatex 1\fi=0 % if pdftex
  \usepackage[T1]{fontenc}
  \usepackage[utf8]{inputenc}
  \usepackage{textcomp} % provide euro and other symbols
\else % if luatex or xetex
  \usepackage{unicode-math}
  \defaultfontfeatures{Scale=MatchLowercase}
  \defaultfontfeatures[\rmfamily]{Ligatures=TeX,Scale=1}
\fi
% Use upquote if available, for straight quotes in verbatim environments
\IfFileExists{upquote.sty}{\usepackage{upquote}}{}
\IfFileExists{microtype.sty}{% use microtype if available
  \usepackage[]{microtype}
  \UseMicrotypeSet[protrusion]{basicmath} % disable protrusion for tt fonts
}{}
\makeatletter
\@ifundefined{KOMAClassName}{% if non-KOMA class
  \IfFileExists{parskip.sty}{%
    \usepackage{parskip}
  }{% else
    \setlength{\parindent}{0pt}
    \setlength{\parskip}{6pt plus 2pt minus 1pt}}
}{% if KOMA class
  \KOMAoptions{parskip=half}}
\makeatother
\usepackage{xcolor}
\IfFileExists{xurl.sty}{\usepackage{xurl}}{} % add URL line breaks if available
\IfFileExists{bookmark.sty}{\usepackage{bookmark}}{\usepackage{hyperref}}
\hypersetup{
  pdftitle={Automation In Daily Life},
  pdfauthor={Rushabh Barbhaya1},
  pdflang={en-EN},
  pdfkeywords={Automation, Jobs, Aviation, Self Driving Cars, Engineering, Science, Production},
  hidelinks,
  pdfcreator={LaTeX via pandoc}}
\urlstyle{same} % disable monospaced font for URLs
\usepackage{graphicx}
\makeatletter
\def\maxwidth{\ifdim\Gin@nat@width>\linewidth\linewidth\else\Gin@nat@width\fi}
\def\maxheight{\ifdim\Gin@nat@height>\textheight\textheight\else\Gin@nat@height\fi}
\makeatother
% Scale images if necessary, so that they will not overflow the page
% margins by default, and it is still possible to overwrite the defaults
% using explicit options in \includegraphics[width, height, ...]{}
\setkeys{Gin}{width=\maxwidth,height=\maxheight,keepaspectratio}
% Set default figure placement to htbp
\makeatletter
\def\fps@figure{htbp}
\makeatother
\setlength{\emergencystretch}{3em} % prevent overfull lines
\providecommand{\tightlist}{%
  \setlength{\itemsep}{0pt}\setlength{\parskip}{0pt}}
\setcounter{secnumdepth}{-\maxdimen} % remove section numbering
% Make \paragraph and \subparagraph free-standing
\ifx\paragraph\undefined\else
  \let\oldparagraph\paragraph
  \renewcommand{\paragraph}[1]{\oldparagraph{#1}\mbox{}}
\fi
\ifx\subparagraph\undefined\else
  \let\oldsubparagraph\subparagraph
  \renewcommand{\subparagraph}[1]{\oldsubparagraph{#1}\mbox{}}
\fi
% Manuscript styling
\usepackage{upgreek}
\captionsetup{font=singlespacing,justification=justified}

% Table formatting
\usepackage{longtable}
\usepackage{lscape}
% \usepackage[counterclockwise]{rotating}   % Landscape page setup for large tables
\usepackage{multirow}		% Table styling
\usepackage{tabularx}		% Control Column width
\usepackage[flushleft]{threeparttable}	% Allows for three part tables with a specified notes section
\usepackage{threeparttablex}            % Lets threeparttable work with longtable

% Create new environments so endfloat can handle them
% \newenvironment{ltable}
%   {\begin{landscape}\centering\begin{threeparttable}}
%   {\end{threeparttable}\end{landscape}}
\newenvironment{lltable}{\begin{landscape}\centering\begin{ThreePartTable}}{\end{ThreePartTable}\end{landscape}}

% Enables adjusting longtable caption width to table width
% Solution found at http://golatex.de/longtable-mit-caption-so-breit-wie-die-tabelle-t15767.html
\makeatletter
\newcommand\LastLTentrywidth{1em}
\newlength\longtablewidth
\setlength{\longtablewidth}{1in}
\newcommand{\getlongtablewidth}{\begingroup \ifcsname LT@\roman{LT@tables}\endcsname \global\longtablewidth=0pt \renewcommand{\LT@entry}[2]{\global\advance\longtablewidth by ##2\relax\gdef\LastLTentrywidth{##2}}\@nameuse{LT@\roman{LT@tables}} \fi \endgroup}

% \setlength{\parindent}{0.5in}
% \setlength{\parskip}{0pt plus 0pt minus 0pt}

% \usepackage{etoolbox}
\makeatletter
\patchcmd{\HyOrg@maketitle}
  {\section{\normalfont\normalsize\abstractname}}
  {\section*{\normalfont\normalsize\abstractname}}
  {}{\typeout{Failed to patch abstract.}}
\patchcmd{\HyOrg@maketitle}
  {\section{\protect\normalfont{\@title}}}
  {\section*{\protect\normalfont{\@title}}}
  {}{\typeout{Failed to patch title.}}
\makeatother
\shorttitle{Automation}
\keywords{Automation, Jobs, Aviation, Self Driving Cars, Engineering, Science, Production\newline\indent Word count: 7}
\DeclareDelayedFloatFlavor{ThreePartTable}{table}
\DeclareDelayedFloatFlavor{lltable}{table}
\DeclareDelayedFloatFlavor*{longtable}{table}
\makeatletter
\renewcommand{\efloat@iwrite}[1]{\immediate\expandafter\protected@write\csname efloat@post#1\endcsname{}}
\makeatother
\usepackage{csquotes}
\ifxetex
  % Load polyglossia as late as possible: uses bidi with RTL langages (e.g. Hebrew, Arabic)
  \usepackage{polyglossia}
  \setmainlanguage[]{english}
\else
  \usepackage[main=english]{babel}
% get rid of language-specific shorthands (see #6817):
\let\LanguageShortHands\languageshorthands
\def\languageshorthands#1{}
\fi
\ifluatex
  \usepackage{selnolig}  % disable illegal ligatures
\fi
\newlength{\cslhangindent}
\setlength{\cslhangindent}{1.5em}
\newlength{\csllabelwidth}
\setlength{\csllabelwidth}{3em}
\newenvironment{CSLReferences}[2] % #1 hanging-ident, #2 entry spacing
 {% don't indent paragraphs
  \setlength{\parindent}{0pt}
  % turn on hanging indent if param 1 is 1
  \ifodd #1 \everypar{\setlength{\hangindent}{\cslhangindent}}\ignorespaces\fi
  % set entry spacing
  \ifnum #2 > 0
  \setlength{\parskip}{#2\baselineskip}
  \fi
 }%
 {}
\usepackage{calc}
\newcommand{\CSLBlock}[1]{#1\hfill\break}
\newcommand{\CSLLeftMargin}[1]{\parbox[t]{\csllabelwidth}{#1}}
\newcommand{\CSLRightInline}[1]{\parbox[t]{\linewidth - \csllabelwidth}{#1}\break}
\newcommand{\CSLIndent}[1]{\hspace{\cslhangindent}#1}

\title{Automation In Daily Life}
\author{Rushabh Barbhaya\textsuperscript{1}}
\date{}


\authornote{

This paper is for the Harrisburg University of Science and Technology's graduate course of Analytics (GRAD 695). This paper uses actual research for generating output. All the research done by this paper can is reproducible. This document is for the internal use of Harrisburg University of Science and Technology and should not be used as a reference in professional study and research.

Correspondence concerning this article should be addressed to Rushabh Barbhaya, 326 Market St, Harrisburg, PA 17101. E-mail: \href{mailto:RBarbhaya@my.harrisburgu.edu}{\nolinkurl{RBarbhaya@my.harrisburgu.edu}}

}

\affiliation{\vspace{0.5cm}\textsuperscript{1} Harrisburg University of Science and Technology}

\abstract{
PlaceHolder
}



\begin{document}
\maketitle

Artificial Intelligence, Machine Learning, Robots, Automation usually form the news as the cause for mass layoffs, for example, Mackie (2021). Author McClure (2018) has similarly observed a correlation between the rise of mainstay automated solutions and growing health concerns. The analysis was not on the level and type of job. This concern for technology replacing jobs has been known and documented since the 16th century. In 1589, William Lee's invention of the stocking machine had caused a riot, noted by Hills (1989) and reported by Fleming (2020). The book ``The Luddites; Machine-Breaking in Regency England,'' authored by Thomis (1972) published in 1972, notes the rise of Luddism. Luddism is a working-class movement asking technology to work with employees and not against them. A modern scripture ``The Digital Divide'' by Nie and Erbring (2001) has a unique perspective on this. Digital divide, which refers to the rift caused by lack of access to information across gender, race, age, among other demographic keys. Nie and Erbring (2001) also says that they see the gap narrowing now. Robinson et al. (2003) pushes findings by Nie and Erbring (2001) a bit further and also notices the bias of information. However, they don't account for future and future technology.

This article by Smith (2019) states that 50\% of Americans believe that Robots will replace innumerable jobs across the industry. An important point is that 80\% believe that their jobs will be secure. It seems counterintuitive, but humans always find a more specialized role, and therefore is not surprising. Acemoglu and Autor (2010) also outlines the same observations. They observed a decline of low-skilled jobs, also raising differences between each level of workers. Computers are replacing jobs where cognitive skills and manual input are obligatory as said by Autor et al. (2003). The author didn't break down the observations across different industrial sectors, where I will be observing the results. Authors also published another article Autor et al. (1998) where they note an increased skill level of an employee in computer-intense industries. This time the author only focused on technology-focused industries and missed out on observing the same trend across other industrial sectors, which is the focus of this research. Humans also fear ``being left behind,'' says Wu Song (2003), and will always try to cover the skills they offset. Illustrated by other papers on this article, we observe a decline in low-skill jobs that are labor-intensive jobs.

\hypertarget{literature-review}{%
\section{Literature Review}\label{literature-review}}

Abernathy and Townsend (1975) observes the evolution of any process. A process that starts as simple logic; evolves into a complex one over time. This complex process generates inefficiencies, and the author points out that those machines are employed to bring back the lost inefficiencies. The author didn't account for how these trends are observed in different industrial sectors. Authors Evangelista and Vezzani (2011) balances out the corporate perspective and speak for human evolution. As robots take on menial jobs, humans find a more specialized role. Those specialized roles spikes growth and knowledge. Similarly, Bainbridge (1982) describes multiple ways in which automation can work in tandem with humans. Humans can take more managerial roles and let machines handle the rule-based task.

\hypertarget{aviation}{%
\subsection{Aviation}\label{aviation}}

At the time of writing, the airline industry is almost automated. Auto-pilot, take-off and landing assistance, navigation, other critical functions are all automated. We still see pilots in the cockpit making sure everything runs smoothly Stanton and Marsden (1996). Berberian et al. (2012) also talks about automation in aviation and also demonstrates that automation decreases responses time and risks. The authors don't dive much into the increasing reliance on technology and converting the human to a checker role. Checking what the robot does and correct it for any issues.

\hypertarget{transportation}{%
\subsection{Transportation}\label{transportation}}

The transportation industry is moving towards automated driving systems. Waymo and Tesla are leading that, among others Rice (2019). They are already saving lives and The Lala et al. (2020) shows that the better the automated systems get, the fewer losses to human lives. The automated systems have already made ways in saving lives as noted by Schwall et al. (2020). These papers, don't talk about what happens if there is an automated car caused an accident. Till driverless cars or self-driving vehicles become a mainstay Ward (2000) proposes the development of an Adaptive Cruise Control system that helps reduce errors and accidents. A need for this cruise control arises because humans have an inherent tendency to make errors as they work on multiple tasks at a time. Having a dedicated machine would help in preventing the loss of lives. The paper doesn't talk about a merger of these technologies.

\hypertarget{manufacturing}{%
\subsection{Manufacturing}\label{manufacturing}}

Of all the industries, the manufacturing industry has utilized robots and artificial intelligence the most. Jämsä-Jounela (2007) talks about how modern industries utilize automation to deliver a reliable product. They use machines anywhere from R\&D to marketing the product. And how each industry utilizes robots. The chemical industry is the biggest one. The authors missed out on the extension of those robotic knowledge/skills to other industrial sectors.

\hypertarget{healthcare}{%
\subsection{Healthcare}\label{healthcare}}

Automation is also taking its place in healthcare with machine learning and AI, outlined by Davenport and Kalakota (2019). This article points out the advances ML and AI have brought to the field. Also points out how a bit value changes and misdiagnose. ML and AI are still evolving in this field and the author(s) believe they will have a major role to play as the models and data evolve. This paper is an overall approach to future possibilities, current use, current limitations, and also live results.

\hypertarget{agriculture}{%
\subsection{Agriculture}\label{agriculture}}

Mahmud et al. (2020) enlighten us about how automation is used in agriculture. Agriculture at a point in history was the only job and now has a very small population engaged in it. Agriculture is probably the space where automation is heavily relied upon for a consistent output. Additionally, Sarangi et al. (2016) demonstrates how automation is used to deal with crop diseases. Mohanraj et al. (2016) talks about how Internet-of-things can be used to yield a better crop with minimum wastage. A farmer wouldn't be able to monitor their farms without additional help. Internet Of Things could help in those cases and notify about any minor change in the field. Also, take measures to avoid harm to the crops. These articles are a good source for understanding how robots and humans can work towards achieving a consistent output and save time.

\hypertarget{future}{%
\subsection{Future}\label{future}}

We are at such a place in the world, where we can deploy another robot to check, and validate the other one. Peleska and Siegel (1996) talk about setting a safety standing for reactive systems. Reactive systems kick in when they see an error and try to correct them. The authors proposed a system, when realized, acts as a check before kicking the reactive system of an automation response of a machine. Although, the authors missed the point of humans checking the robot's checked work. Making sure that there are no false positives and false negatives in the response. Daily et al. (2017) looks at how, when a machine is released in the real world would be affected by 3 things. 1. Government regulation, 2. Interference of historical perception to new technologies implementation and 3. Future. The author missed adding public acceptance of technology. There are a lot of unknowns but in the end, humans always accept machines as they are convenient and safe. Badue et al. (2021) tests out how each self-driving car's system operates and functions. All the functions they test were industry standard. Most of the functions of each machine were hidden from the authors but safety standards were always maintained as per their independent testing.\\
Greenblatt (2016) suggests a hypothetical scenario for self-driving cars and a potential lawsuit. The authors leave an open-ended question after walking through each of the scenarios. The end goal of this exercise is to answer the question, who is to blame when technology is involved in an accident with humans. Strawn (2016) describes an open-ended question, to what happens when the future is completely automated. Will it cause a utopia or a dystopia. Proving sound arguments on both ends.

\hypertarget{hypothesis}{%
\subsection{Hypothesis}\label{hypothesis}}

\hypertarget{primary-hypothesis}{%
\subsubsection{Primary hypothesis}\label{primary-hypothesis}}

Process automation in the finance industry will result in job losses
The primary hypothesis will be an extrapolated result from the other hypothesis.

\hypertarget{hypothesis-1}{%
\subsubsection{Hypothesis 1}\label{hypothesis-1}}

Automated Landing and Piloting systems in the aviation industry results in job loss.

\hypertarget{hypothesis-2}{%
\subsubsection{Hypothesis 2}\label{hypothesis-2}}

Automated Landing and Piloting systems will raise human safety concerns.

\hypertarget{hypothesis-3}{%
\subsubsection{Hypothesis 3}\label{hypothesis-3}}

Self-driving and Driverless cars will raise human safety concerns.

\hypertarget{method}{%
\section{Method}\label{method}}

Using the data provided by the United States Bureau Of Labor Statistics ({``Featured CES Searchable Databases''} (n.d.).). The data is sliced by year and industrial section. US Bureau of Labor Statistics houses data in multiple formats, using simple flat files for this analysis. US Bureau of Labor Statistics also provides with data key to decode the column names. Similarly, safety data is also obtained from government agencies. IATA {``Safety Data''} (n.d.) provides a safety report across years for this analysis. Department of Transportation provides road safety reports. Data will be combined from official government-managed websites and other sources that confirm when automation was implemented in that industry. For example, Auto-Pilot was invented in 1912 Windsor (1913) with its initial implementation in 1970 following the Lunar Landing Module Widnall (1971).\\
Following the results from these sections, the results are trend analyzed and extrapolated to the finance industry to conclude the suggested hypothesis.

\hypertarget{participants}{%
\subsection{Participants}\label{participants}}

Absolute numbers from the statistics reports as the participants in this analysis. These reports are ranged from 1930 to 2020. The key column in this report being the unemployment rate across industries and safety reports.\\
The reports and articles from multiple sources will be utilized to check if a robotic process resulted in a loss of jobs and loss of lives. Or was it the loss of lives and loss of jobs that gave rise to robots?

\hypertarget{procedures}{%
\subsection{Procedures}\label{procedures}}

The first step in this analysis is to clean, treat outliers and normalize the data. Cleaning the data entails checking for any formatting issues, removing or excluding any unwanted data that impacts the speed of execution. A correctly formatting the data to the correct data type that is used in the analysis. Converting percent to 0-1 normalized values.\\
Outliers are subjective. Outliers affect the final results of the analysis. Outliers may skew the results in any direction; therefore, it becomes essential to identify and treat them accordingly.\\
It is essential to normalize data that spans multiple years to a joint base. Treating safety reports and employment records to parts per thousand is the first step before any analysis is performed. This makes comparison on equal terms.\\
Following these steps, z-test, and analysis of variance to conclude the proposed hypothesis.\\

\hypertarget{measures}{%
\subsection{Measures}\label{measures}}

For the hypothesis tests, the confidence interval is set at 90\% as there are for almost a century which may result in a more significant change. The measure of success is calculated on p-values from the hypothesis testing with an acceptable p-value of less than or equal to 0.01 and significant variables from the analysis.\\
These results are weighted and applied to the finance industry, which is the basis for this research. The weighing factor is ranged from 0 to 1 and is fed to a Monte Carlo series to account for any random error which may occur.

\hypertarget{analysis}{%
\subsection{Analysis}\label{analysis}}

The overall mean from the Monte Carlo simulation results is compared with the trends in technology, aviation, manufacturing, and agriculture industries. This result will not only give the current position on the automation trend but also give an estimate. Estimation of how long it would take for robots to be accepted in this industry. Monte Carlo series helps normalize and accounts for any random error that may occur.\\
R (R Core Team (2021)) is used to extract all the results, except the Monte Carlo simulation. Monte Carlo simulation is performed by python ({``Python Release Python 3.9.7''} (n.d.)).

\newpage

\hypertarget{references}{%
\section{References}\label{references}}

\begingroup
\setlength{\parindent}{-0.5in}
\setlength{\leftskip}{0.5in}

\hypertarget{refs}{}
\begin{CSLReferences}{1}{0}
\leavevmode\hypertarget{ref-ABERNATHY1975379}{}%
Abernathy, W. J., \& Townsend, P. L. (1975). Technology, productivity and process change. \emph{Technological Forecasting and Social Change}, \emph{7}(4), 379--396. https://doi.org/\url{https://doi.org/10.1016/0040-1625(75)90015-3}

\leavevmode\hypertarget{ref-NBERw16082}{}%
Acemoglu, D., \& Autor, D. (2010). \emph{Skills, tasks and technologies: Implications for employment and earnings} (Working Paper No. 16082; Working Paper Series). National Bureau of Economic Research. \url{https://doi.org/10.3386/w16082}

\leavevmode\hypertarget{ref-10.1162ux2f003355398555874}{}%
Autor, D. H., Katz, L. F., \& Krueger, A. B. (1998). {Computing Inequality: Have Computers Changed the Labor Market?*}. \emph{The Quarterly Journal of Economics}, \emph{113}(4), 1169--1213. \url{https://doi.org/10.1162/003355398555874}

\leavevmode\hypertarget{ref-10.1162ux2f003355303322552801}{}%
Autor, D. H., Levy, F., \& Murnane, R. J. (2003). {The Skill Content of Recent Technological Change: An Empirical Exploration*}. \emph{The Quarterly Journal of Economics}, \emph{118}(4), 1279--1333. \url{https://doi.org/10.1162/003355303322552801}

\leavevmode\hypertarget{ref-badue2021self}{}%
Badue, C., Guidolini, R., Carneiro, R. V., Azevedo, P., Cardoso, V. B., Forechi, A., Jesus, L., Berriel, R., Paixao, T. M., Mutz, F., \& others. (2021). Self-driving cars: A survey. \emph{Expert Systems with Applications}, \emph{165}, 113816.

\leavevmode\hypertarget{ref-bainbridge1982ironies}{}%
Bainbridge, L. (1982). Ironies of automation. \emph{IFAC Proceedings Volumes}, \emph{15}(6), 129--135.

\leavevmode\hypertarget{ref-2247952820120101}{}%
Berberian, B., Sarrazin, J.-C., Le Blaye, P., \& Haggard, P. (2012). Automation technology and sense of control: A window on human agency. \emph{PloS One}, \emph{7}(3), e34075. \url{https://search.ebscohost.com/login.aspx?direct=true\&AuthType=sso\&db=mdc\&AN=22479528\&site=ehost-live\&scope=site\&custid=s4786267}

\leavevmode\hypertarget{ref-daily2017self}{}%
Daily, M., Medasani, S., Behringer, R., \& Trivedi, M. (2017). Self-driving cars. \emph{Computer}, \emph{50}(12), 18--23.

\leavevmode\hypertarget{ref-davenport2019potential}{}%
Davenport, T., \& Kalakota, R. (2019). The potential for artificial intelligence in healthcare. \emph{Future Healthcare Journal}, \emph{6}(2), 94.

\leavevmode\hypertarget{ref-10.1093ux2ficcux2fdtr069}{}%
Evangelista, R., \& Vezzani, A. (2011). {The impact of technological and organizational innovations on employment in European firms}. \emph{Industrial and Corporate Change}, \emph{21}(4), 871--899. \url{https://doi.org/10.1093/icc/dtr069}

\leavevmode\hypertarget{ref-u.s}{}%
Featured CES searchable databases. (n.d.). In \emph{U.S. Bureau of Labor Statistics}. U.S. Bureau of Labor Statistics. \url{https://www.bls.gov/ces/data/\#}

\leavevmode\hypertarget{ref-fleming_2020}{}%
Fleming, S. (2020). A short history of jobs and automation. In \emph{World Economic Forum}. \url{https://www.weforum.org/agenda/2020/09/short-history-jobs-automation/}

\leavevmode\hypertarget{ref-7419800}{}%
Greenblatt, N. A. (2016). Self-driving cars and the law. \emph{IEEE Spectrum}, \emph{53}(2), 46--51. \url{https://doi.org/10.1109/MSPEC.2016.7419800}

\leavevmode\hypertarget{ref-hills1989william}{}%
Hills, R. (1989). William lee and his knitting machine. \emph{Journal of the Textile Institute}, \emph{80}(2), 169--184.

\leavevmode\hypertarget{ref-jamsa2007future}{}%
Jämsä-Jounela, S.-L. (2007). Future trends in process automation. \emph{Annual Reviews in Control}, \emph{31}(2), 211--220.

\leavevmode\hypertarget{ref-10.1145ux2f3411053}{}%
Lala, J. H., Landwehr, C. E., \& Meyer, J. F. (2020). Autonomous vehicle safety: Lessons from aviation. \emph{Commun. ACM}, \emph{63}(9), 28--31. \url{https://doi.org/10.1145/3411053}

\leavevmode\hypertarget{ref-mackie_2021}{}%
Mackie, C. (2021). Beware professional services workers: Robots are coming for your job too! In \emph{Forbes}. Forbes Magazine. \url{https://www.forbes.com/sites/calvinmackie/2021/09/30/beware-professional-services-workers-robots-are-coming-for-your-job-too/?sh=44a36ca25237}

\leavevmode\hypertarget{ref-mahmud2020robotics}{}%
Mahmud, M. S. A., Abidin, M. S. Z., Emmanuel, A. A., \& Hasan, H. S. (2020). Robotics and automation in agriculture: Present and future applications. \emph{Applications of Modelling and Simulation}, \emph{4}, 130--140.

\leavevmode\hypertarget{ref-PaulKMcclure2018}{}%
McClure, P. K. (2018). {``You're fired,''} says the robot: The rise of automation in the workplace, technophobes, and fears of unemployment. \emph{Social Science Computer Review}, \emph{36}(2), 139--156. \url{https://doi.org/10.1177/0894439317698637}

\leavevmode\hypertarget{ref-mohanraj2016field}{}%
Mohanraj, I., Ashokumar, K., \& Naren, J. (2016). Field monitoring and automation using IOT in agriculture domain. \emph{Procedia Computer Science}, \emph{93}, 931--939.

\leavevmode\hypertarget{ref-10.5555ux2f762625.762640}{}%
Nie, N. H., \& Erbring, L. (2001). Internet and society: A preliminary report. In \emph{The digital divide: Facing a crisis or creating a myth?} (pp. 269--271). MIT Press.

\leavevmode\hypertarget{ref-peleska1996test}{}%
Peleska, J., \& Siegel, M. (1996). \emph{Test automation of safety-critical reactive systems}.

\leavevmode\hypertarget{ref-python.org}{}%
Python release python 3.9.7. (n.d.). In \emph{Python.org}. \url{https://www.python.org/downloads/release/python-397/}

\leavevmode\hypertarget{ref-R-base}{}%
R Core Team. (2021). \emph{R: A language and environment for statistical computing}. R Foundation for Statistical Computing. \url{https://www.R-project.org/}

\leavevmode\hypertarget{ref-13645118320190101}{}%
Rice, D. (2019). The driverless car and the legal system: Hopes and fears as the courts, regulatory agencies, waymo, tesla, and uber deal with this exciting and terrifying new technology. \emph{Journal of Strategic Innovation \& Sustainability}, \emph{14}(1), 134--146. \url{https://search.ebscohost.com/login.aspx?direct=true\&AuthType=sso\&db=sur\&AN=136451183\&site=ehost-live\&scope=site\&custid=s4786267}

\leavevmode\hypertarget{ref-robinson2003new}{}%
Robinson, J. P., DiMaggio, P., \& Hargittai, E. (2003). New social survey perspectives on the digital divide. \emph{It \& Society}, \emph{1}(5), 1--22.

\leavevmode\hypertarget{ref-iata}{}%
Safety data. (n.d.). In \emph{IATA}. \url{https://www.iata.org/en/services/statistics/safety-data/}

\leavevmode\hypertarget{ref-sarangi2016automation}{}%
Sarangi, S., Umadikar, J., \& Kar, S. (2016). Automation of agriculture support systems using wisekar: Case study of a crop-disease advisory service. \emph{Computers and Electronics in Agriculture}, \emph{122}, 200--210.

\leavevmode\hypertarget{ref-schwall2020waymo}{}%
Schwall, M., Daniel, T., Victor, T., Favaro, F., \& Hohnhold, H. (2020). Waymo public road safety performance data. \emph{arXiv Preprint arXiv:2011.00038}.

\leavevmode\hypertarget{ref-smith_2019}{}%
Smith, A. (2019). \emph{Methodology}. \url{https://www.pewresearch.org/internet/2016/03/10/future-of-workforce-automation-methodology/}

\leavevmode\hypertarget{ref-STANTON199635}{}%
Stanton, N. A., \& Marsden, P. (1996). From fly-by-wire to drive-by-wire: Safety implications of automation in vehicles. \emph{Safety Science}, \emph{24}(1), 35--49. https://doi.org/\url{https://doi.org/10.1016/S0925-7535(96)00067-7}

\leavevmode\hypertarget{ref-strawn2016automation}{}%
Strawn, G. (2016). Automation and future unemployment. \emph{IT Professional}, \emph{18}(1), 62--64.

\leavevmode\hypertarget{ref-thomis1972luddites}{}%
Thomis, M. I. (1972). \emph{The luddites; machine-breaking in regency england}. Schocken Books Incorporated.

\leavevmode\hypertarget{ref-1336134720000101}{}%
Ward, N. J. (2000). Automation of task processes: An example of intelligent transportation systems. \emph{Human Factors \& Ergonomics in Manufacturing}, \emph{10}(4), 395--408. \url{https://search.ebscohost.com/login.aspx?direct=true\&AuthType=sso\&db=asn\&AN=13361347\&site=ehost-live\&scope=site\&custid=s4786267}

\leavevmode\hypertarget{ref-widnall1971lunar}{}%
Widnall, W. S. (1971). Lunar module digital autopilot. \emph{Journal of Spacecraft and Rockets}, \emph{8}(1), 56--62.

\leavevmode\hypertarget{ref-windsor1913popular}{}%
Windsor, H. H. (1913). \emph{Popular mechanics magazine} (Vol. 20). Popular Mechanics Company.

\leavevmode\hypertarget{ref-1324077020030101}{}%
Wu Song, F. (2003). Being left behind: The discourse of fear in technological change. \emph{Hedgehog Review}, \emph{5}(3), 26--42. \url{https://search.ebscohost.com/login.aspx?direct=true\&AuthType=sso\&db=hus\&AN=13240770\&site=ehost-live\&scope=site\&custid=s4786267}

\end{CSLReferences}

\endgroup


\end{document}
